% Options for packages loaded elsewhere
\PassOptionsToPackage{unicode}{hyperref}
\PassOptionsToPackage{hyphens}{url}
%
\documentclass[
]{article}
\usepackage{amsmath,amssymb}
\usepackage{lmodern}
\usepackage{ifxetex,ifluatex}
\ifnum 0\ifxetex 1\fi\ifluatex 1\fi=0 % if pdftex
  \usepackage[T1]{fontenc}
  \usepackage[utf8]{inputenc}
  \usepackage{textcomp} % provide euro and other symbols
\else % if luatex or xetex
  \usepackage{unicode-math}
  \defaultfontfeatures{Scale=MatchLowercase}
  \defaultfontfeatures[\rmfamily]{Ligatures=TeX,Scale=1}
\fi
% Use upquote if available, for straight quotes in verbatim environments
\IfFileExists{upquote.sty}{\usepackage{upquote}}{}
\IfFileExists{microtype.sty}{% use microtype if available
  \usepackage[]{microtype}
  \UseMicrotypeSet[protrusion]{basicmath} % disable protrusion for tt fonts
}{}
\makeatletter
\@ifundefined{KOMAClassName}{% if non-KOMA class
  \IfFileExists{parskip.sty}{%
    \usepackage{parskip}
  }{% else
    \setlength{\parindent}{0pt}
    \setlength{\parskip}{6pt plus 2pt minus 1pt}}
}{% if KOMA class
  \KOMAoptions{parskip=half}}
\makeatother
\usepackage{xcolor}
\IfFileExists{xurl.sty}{\usepackage{xurl}}{} % add URL line breaks if available
\IfFileExists{bookmark.sty}{\usepackage{bookmark}}{\usepackage{hyperref}}
\hypersetup{
  pdftitle={Getting around GitHub Participants},
  pdfauthor={Michelle Slawinski},
  hidelinks,
  pdfcreator={LaTeX via pandoc}}
\urlstyle{same} % disable monospaced font for URLs
\usepackage[margin=1in]{geometry}
\usepackage{graphicx}
\makeatletter
\def\maxwidth{\ifdim\Gin@nat@width>\linewidth\linewidth\else\Gin@nat@width\fi}
\def\maxheight{\ifdim\Gin@nat@height>\textheight\textheight\else\Gin@nat@height\fi}
\makeatother
% Scale images if necessary, so that they will not overflow the page
% margins by default, and it is still possible to overwrite the defaults
% using explicit options in \includegraphics[width, height, ...]{}
\setkeys{Gin}{width=\maxwidth,height=\maxheight,keepaspectratio}
% Set default figure placement to htbp
\makeatletter
\def\fps@figure{htbp}
\makeatother
\setlength{\emergencystretch}{3em} % prevent overfull lines
\providecommand{\tightlist}{%
  \setlength{\itemsep}{0pt}\setlength{\parskip}{0pt}}
\setcounter{secnumdepth}{-\maxdimen} % remove section numbering
\ifluatex
  \usepackage{selnolig}  % disable illegal ligatures
\fi

\title{Getting around GitHub Participants}
\author{Michelle Slawinski}
\date{3/6/2022}

\begin{document}
\maketitle

We will use this file to not only document who attended the Getting
around GitHub tutorial but as a practice document for how to grab a
repository, make changes to a file, commit those changes and push them
back to the main repository on GitHub.

Below is how you code a list in an RMarkdown file. Type your name on one
of the following lines.

\begin{enumerate}
\def\labelenumi{\arabic{enumi}.}
\tightlist
\item
  Michelle Slawinski
\item
\item
\item
\item
\item
\item
\item
\item
\item
\item
\item
\item
\item
\item
\end{enumerate}

Next, save your changes. Then you can either use RStudio or GitHub
Desktop to commit and push your changes, your choice! It will depend on
what you have installed and set up.

\hypertarget{github}{%
\subsection{GitHub}\label{github}}

If you are using GitHub to commit and push your changes, then you should
have it downloaded on your computer.

\begin{enumerate}
\def\labelenumi{\arabic{enumi}.}
\tightlist
\item
  Open GitHub Desktop
\item
  Click the ``Current Repository'' in the upper lefthand corner
\item
  Click the ``practice'' repository
\item
  If you saved your changes in RStudio, you should see that there is an
  orange box with a dot inside. If you hover over this, it should say
  ``modified''
\item
  Below this, there is a text box. This is where you will type a commit
  message. Remember, committing just means you are saving this to your
  local operating system. The message should be short but descriptive.
  One tip is to keep your message in present tense. An easy way to think
  about it is imagine you start each commit with, ``This
  commit\ldots{}'' then your message.
\item
  Example: Updates the participant list with my name.
\item
  Adds `Michelle Slawinski' to list.
\item
  Press commit!
\item
  Now, you have saved your commit locally, but we want to push this to
  the remote repository on GitHub where the author can accept your
  changes. Press ``Push.''
\end{enumerate}

\hypertarget{rstudio}{%
\subsection{RStudio}\label{rstudio}}

If you are using RStudio to commit and push your changes, then you
should have already connecting Git and RStudio. If you did, you will see
the git symbol up in the menu bar.

\begin{enumerate}
\def\labelenumi{\arabic{enumi}.}
\tightlist
\item
  Click the git symbol and press ``commit''
\item
  Stage (i.e., select) the files that you want to save.
\item
  Write a commit message in the description box. Like with GitHub
  desktop, you will want to use present tense. See above.
\item
  Press commit!
\item
  Then click ``Push'' and wallah!
\end{enumerate}

\end{document}
